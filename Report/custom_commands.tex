\usepackage{amsfonts}
\usepackage{dsfont}
\usepackage{amssymb}
\usepackage{graphicx}
\usepackage{subcaption}
\usepackage{titlesec}
\usepackage{appendix}
\usepackage{amsmath}
\usepackage{pgfplots}
\usepackage{tikz}
\usepackage{stmaryrd}
\usepackage{circuitikz}
\usepackage{cancel}
\usepackage{slashed}
\usepackage{appendix}
\usepackage{hyperref}
\usepackage{siunitx}
\usepackage{url}
\usepackage[most]{tcolorbox}
\usepackage[margin=0.7in]{geometry}
\usepackage[compat=1.1.0]{tikz-feynman}
\usepackage[overload]{empheq}
\usepackage{braket}
\usepackage{mathtools}
\usepackage{float}
\usepackage{lineno}
\usepackage{xargs}
\usepackage{svg}


% better underbrace ==============================================

\usetikzlibrary{decorations.pathreplacing}
\makeatletter
\def\underbrace#1{%
	\@ifnextchar_{\tikz@@underbrace{#1}}{\tikz@@underbrace{#1}_{}}}
\def\tikz@@underbrace#1_#2{%
	\tikz[baseline=(a.base)] {\node[inner sep=0] (a) {\(\displaystyle #1\)};
		\draw[line cap=round,decorate,decoration={brace,amplitude=5pt}]
		(a.south east) -- node[below,inner sep=7pt] {\(\scriptstyle #2\)} (a.south west);}}
\def\overbrace#1{%
	\@ifnextchar^{\tikz@@overbrace{#1}}{\tikz@@overbrace{#1}^{}}}
\def\tikz@@overbrace#1^#2{%
	\tikz[baseline=(a.base)] {\node[inner sep=0] (a) {\(#1\)};
		\draw[line cap=round,decorate,decoration={brace,amplitude=5pt}]
		(a.north west) -- node[pos=.5,above,inner sep=7pt] {\(\scriptstyle #2\)} (a.north east);}}
\makeatother


% color box ==============================================================
\usepackage{xcolor}

% Syntax: \colorboxed[<color model>]{<color specification>}{<math formula>}
\newcommand*{\colorboxed}{}
\def\colorboxed#1#{%
	\colorboxedAux{#1}%
}
\newcommand*{\colorboxedAux}[3]{%
	% #1: optional argument for color model
	% #2: color specification
	% #3: formula
	\begingroup
	\colorlet{cb@saved}{.}%
	\color#1{#2}%
	\boxed{%
		\color{cb@saved}%
		#3%
	}%
	\endgroup
}

% miscellaneous commands

\newcommand{\dd}[2]{\frac{\mathrm{d}#1}{\mathrm{d}#2}}
\newcommand{\dpart}[2]{\frac{\partial#1}{\partial#2}}
\newcommandx{\intg}[4][2=,3=,4=]{\int\IfValueT{#3}{_{#3}}\IfValueT{#4}{^{#4}}\mathrm{d}\IfValueT{#2}{^{#2}}#1\hspace{1mm}}
\newcommand{\feynint}[1]{\int\frac{\mathrm{d}^D#1}{\left(2\pi\right)^D}}
\newcommand{\LL}{\mathrm{L}}
\newcommand{\RR}{\mathrm{R}}