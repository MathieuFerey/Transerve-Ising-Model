\documentclass[11pt,openany]{article}

\usepackage{amsfonts}
\usepackage{dsfont}
\usepackage{amssymb}
\usepackage{graphicx}
\usepackage{subcaption}
\usepackage{titlesec}
\usepackage{appendix}
\usepackage{amsmath}
\usepackage{pgfplots}
\usepackage{tikz}
\usepackage{stmaryrd}
\usepackage{circuitikz}
\usepackage{cancel}
\usepackage{slashed}
\usepackage{appendix}
\usepackage{hyperref}
\usepackage{siunitx}
\usepackage{url}
\usepackage[most]{tcolorbox}
\usepackage[margin=0.7in]{geometry}
\usepackage[compat=1.1.0]{tikz-feynman}
\usepackage[overload]{empheq}
\usepackage{braket}
\usepackage{mathtools}
\usepackage{float}
\usepackage{lineno}
\usepackage{xargs}
\usepackage{svg}


% better underbrace ==============================================

\usetikzlibrary{decorations.pathreplacing}
\makeatletter
\def\underbrace#1{%
	\@ifnextchar_{\tikz@@underbrace{#1}}{\tikz@@underbrace{#1}_{}}}
\def\tikz@@underbrace#1_#2{%
	\tikz[baseline=(a.base)] {\node[inner sep=0] (a) {\(\displaystyle #1\)};
		\draw[line cap=round,decorate,decoration={brace,amplitude=5pt}]
		(a.south east) -- node[below,inner sep=7pt] {\(\scriptstyle #2\)} (a.south west);}}
\def\overbrace#1{%
	\@ifnextchar^{\tikz@@overbrace{#1}}{\tikz@@overbrace{#1}^{}}}
\def\tikz@@overbrace#1^#2{%
	\tikz[baseline=(a.base)] {\node[inner sep=0] (a) {\(#1\)};
		\draw[line cap=round,decorate,decoration={brace,amplitude=5pt}]
		(a.north west) -- node[pos=.5,above,inner sep=7pt] {\(\scriptstyle #2\)} (a.north east);}}
\makeatother


% color box ==============================================================
\usepackage{xcolor}

% Syntax: \colorboxed[<color model>]{<color specification>}{<math formula>}
\newcommand*{\colorboxed}{}
\def\colorboxed#1#{%
	\colorboxedAux{#1}%
}
\newcommand*{\colorboxedAux}[3]{%
	% #1: optional argument for color model
	% #2: color specification
	% #3: formula
	\begingroup
	\colorlet{cb@saved}{.}%
	\color#1{#2}%
	\boxed{%
		\color{cb@saved}%
		#3%
	}%
	\endgroup
}

% miscellaneous commands

\newcommand{\dd}[2]{\frac{\mathrm{d}#1}{\mathrm{d}#2}}
\newcommand{\dpart}[2]{\frac{\partial#1}{\partial#2}}
\newcommandx{\intg}[4][2=,3=,4=]{\int\IfValueT{#3}{_{#3}}\IfValueT{#4}{^{#4}}\mathrm{d}\IfValueT{#2}{^{#2}}#1\hspace{1mm}}
\newcommand{\feynint}[1]{\int\frac{\mathrm{d}^D#1}{\left(2\pi\right)^D}}
\newcommand{\LL}{\mathrm{L}}
\newcommand{\RR}{\mathrm{R}}

\setlength\parindent{0pt}

\title{The Transverse Ising Model }
\author{Mathieu Ferey}


\begin{document}
	
\begin{titlepage}
    
    \begin{center}
        ETH Zürich
        
        \vspace{1cm}

        \Huge
        \textbf{Quantum Simulations\\\huge of Gauge Theories\\}
        
        \vspace{1.5cm}

        \Large
        High Energy Physics Master Program\\
        Mathieu FEREY 
        
        \begin{figure}[h]
            \centering 
            \includegraphics[scale=0.2]{Images/hep-logo.png}
        \end{figure}
        
        \rule{13cm}{0.5mm}
        \huge
        \textbf{\\Pure $\mathds{Z}_2$ Gauge Theory\\}
        \vspace{2mm}
        \Large
        \textbf{and the Transverse Field Ising Model\\}
        \rule{13cm}{0.5mm}
        
        \vspace{5mm}
        Herbstsemester 2023
        
        \vspace{1cm}
        
        \begin{figure}[h]
            \centering
            \includegraphics[width=0.5\textwidth]{Images/eth_logo.png}
        \end{figure}
    
    	\vfill
    	
	    \large
		\underline{Lecturers}
	
		\vspace{1mm}
		Prof. Dr. Marina Krstic Marinkovic\\
		Dr. Joao Carlos Pinto Barros\\
		
		\vspace{1cm}
        \small
        Institut für Theoretische Physik\\
        ETH Zürich\\
        

    \end{center}


\end{titlepage}

\section{Introduction}


The Ising gauge theory, theory with a discrete $\mathds{Z}_2$ gauge symmetry in $2+1$-dimensions

\begin{equation}
	H_{\mathds{Z}_2} = -g\sum_{\vec{x},j}\sigma^x_j(\vec{x}) - \frac{1}{g}\sum_{\vec{x}}\sigma_1^z(\vec{x})\sigma_2^z(\vec{x}+e_1)\sigma_1^z(\vec{x}+e_2)\sigma_2^z(\vec{x}),
\end{equation}

where $\vec{x}$ refers to a position on the lattice, $j=1,2$ the two possible directions of a link, $\sigma_j^{x/z}(\vec{x})$ are the Pauli matrices, living on the links of the lattice. The local operator

\begin{equation}
	Q(\vec{x}) \equiv \sigma_1^x(\vec{x})\sigma_1^x(\vec{x}-e_1)\sigma_2^x(\vec{x})\sigma_2^x(\vec{x}-e_2)
\end{equation}

commutes with $H_{\mathds{Z}_2}$. It generates local gauge transformations. One can check that $Q^2=\mathds{1}$, so that the local symmetry of our problem is indeed $\mathds{Z}_2$. The operator

\begin{equation}
	\tau^z(\vec{r}) = \prod_{(\vec{x},j) \text{ pierced by } \gamma(\vec{r})}\sigma_j^x(\vec{x}),
\end{equation}

called the magnetic charge, is a gauge invariant quantity. $\gamma$ is an open path on the dual lattice. Since

\begin{equation}
	\left\{W_p^2(\vec{r}),\tau_z^2(\vec{r})\right\} = 0\text{ and }W_p^2(\vec{r}) = \tau_z^2(\vec{r}) = \mathds{1},
\end{equation} 

one can identify $W_p$ with the Pauli matrice $\tau^x$ on  the dual lattice.

\begin{equation}
	H = -\sum_{i=1}^N\tau^z_i\tau^z_{i+1} - g\sum\tau^x_j.
\end{equation}

where $i,j$ run over the dual lattice sites.

Map it to a $d+1$-dim classical anisotropic Ising Model

\begin{equation}
	H_\mathrm{class} = -\frac{N_y\gamma}{\beta}\sum_{i=1}^N\sum_{j=1}^{N_y}\sigma_z^{(i,j)}\sigma_z^{(i,j+1)} - \sum_{i=1}^N\sum_{j=1}^{N_y}\sigma_z^{(i,j)}\sigma_z^{(i+1,j)},
\end{equation}

with $\gamma = -\frac{1}{2}\log\tanh{a}$, $a = \frac{-\beta g}{N_y}$.

In general

\begin{equation}
	H = -J\sum_{i,j}S_i^zS_j^z - \Gamma\sum_iS_i^x,
\end{equation}

can be mapped to

\begin{equation}
	H_\mathrm{eff}(M) = -\sum_{k=1}^M\left[\frac{K}{M}\sum_{i,j}S_{ik}S_{jk} + K_M\sum_iS_{ik}S_{ik+1}\right],
\end{equation}

with $K_M = \dfrac{1}{2}\ln\left(\coth\left(\beta\Gamma/M\right)\right)$ and $K=J\beta$.

\section{MCMC-MH}


\begin{tcolorbox}[title=Metropolis-Hasting algorithm]
	
	\begin{enumerate}
		
		\item Select initial value $\theta_0$.
		
		\item For $i\in\left\{1,\cdots,N_\mathrm{sample}\right\}$:
			\begin{itemize}
				\item Draw candidate $\theta^*$ from proposal distribution $q(\theta^*|\theta_{i-1})$.
				
				\item Compute $\alpha = \dfrac{g(\theta^*)}{g(\theta_{i-1})}\dfrac{q(\theta_{i-1}|\theta^*)}{q(\theta^*|\theta_{i-1})}$.
				
				\item If $\alpha \geq 1$ accept $\theta^*$ by setting $\theta_i\leftarrow\theta^*$,\\
				If $0 < \alpha <1$ accept $\theta^*$ with probability $\alpha$.
			\end{itemize}
		
	\end{enumerate}
	
\end{tcolorbox}




	
\end{document}