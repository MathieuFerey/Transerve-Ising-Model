\documentclass[11pt,openany]{article}

\usepackage{amsfonts}
\usepackage{dsfont}
\usepackage{amssymb}
\usepackage{graphicx}
\usepackage{subcaption}
\usepackage{titlesec}
\usepackage{appendix}
\usepackage{amsmath}
\usepackage{pgfplots}
\usepackage{tikz}
\usepackage{stmaryrd}
\usepackage{circuitikz}
\usepackage{cancel}
\usepackage{slashed}
\usepackage{appendix}
\usepackage{hyperref}
\usepackage{siunitx}
\usepackage{url}
\usepackage[most]{tcolorbox}
\usepackage[margin=0.7in]{geometry}
\usepackage[compat=1.1.0]{tikz-feynman}
\usepackage[overload]{empheq}
\usepackage{braket}
\usepackage{mathtools}
\usepackage{float}
\usepackage{lineno}
\usepackage{xargs}
\usepackage{svg}


% better underbrace ==============================================

\usetikzlibrary{decorations.pathreplacing}
\makeatletter
\def\underbrace#1{%
	\@ifnextchar_{\tikz@@underbrace{#1}}{\tikz@@underbrace{#1}_{}}}
\def\tikz@@underbrace#1_#2{%
	\tikz[baseline=(a.base)] {\node[inner sep=0] (a) {\(\displaystyle #1\)};
		\draw[line cap=round,decorate,decoration={brace,amplitude=5pt}]
		(a.south east) -- node[below,inner sep=7pt] {\(\scriptstyle #2\)} (a.south west);}}
\def\overbrace#1{%
	\@ifnextchar^{\tikz@@overbrace{#1}}{\tikz@@overbrace{#1}^{}}}
\def\tikz@@overbrace#1^#2{%
	\tikz[baseline=(a.base)] {\node[inner sep=0] (a) {\(#1\)};
		\draw[line cap=round,decorate,decoration={brace,amplitude=5pt}]
		(a.north west) -- node[pos=.5,above,inner sep=7pt] {\(\scriptstyle #2\)} (a.north east);}}
\makeatother


% color box ==============================================================
\usepackage{xcolor}

% Syntax: \colorboxed[<color model>]{<color specification>}{<math formula>}
\newcommand*{\colorboxed}{}
\def\colorboxed#1#{%
	\colorboxedAux{#1}%
}
\newcommand*{\colorboxedAux}[3]{%
	% #1: optional argument for color model
	% #2: color specification
	% #3: formula
	\begingroup
	\colorlet{cb@saved}{.}%
	\color#1{#2}%
	\boxed{%
		\color{cb@saved}%
		#3%
	}%
	\endgroup
}

% miscellaneous commands

\newcommand{\dd}[2]{\frac{\mathrm{d}#1}{\mathrm{d}#2}}
\newcommand{\dpart}[2]{\frac{\partial#1}{\partial#2}}
\newcommandx{\intg}[4][2=,3=,4=]{\int\IfValueT{#3}{_{#3}}\IfValueT{#4}{^{#4}}\mathrm{d}\IfValueT{#2}{^{#2}}#1\hspace{1mm}}
\newcommand{\feynint}[1]{\int\frac{\mathrm{d}^D#1}{\left(2\pi\right)^D}}
\newcommand{\LL}{\mathrm{L}}
\newcommand{\RR}{\mathrm{R}}

\setlength\parindent{0pt}

\title{The Transverse Ising Model }
\author{Mathieu Ferey}

\addbibresource{biblio.bib}


\begin{document}
	
\begin{titlepage}
    
    \begin{center}
        ETH Zürich
        
        \vspace{1cm}

        \Huge
        \textbf{Quantum Simulations\\\huge of Gauge Theories\\}
        
        \vspace{1.5cm}

        \Large
        High Energy Physics Master Program\\
        Mathieu FEREY 
        
        \begin{figure}[h]
            \centering 
            \includegraphics[scale=0.2]{Images/hep-logo.png}
        \end{figure}
        
        \rule{13cm}{0.5mm}
        \huge
        \textbf{\\Pure $\mathds{Z}_2$ Gauge Theory\\}
        \vspace{2mm}
        \Large
        \textbf{and the Transverse Field Ising Model\\}
        \rule{13cm}{0.5mm}
        
        \vspace{5mm}
        Herbstsemester 2023
        
        \vspace{1cm}
        
        \begin{figure}[h]
            \centering
            \includegraphics[width=0.5\textwidth]{Images/eth_logo.png}
        \end{figure}
    
    	\vfill
    	
	    \large
		\underline{Lecturers}
	
		\vspace{1mm}
		Prof. Dr. Marina Krstic Marinkovic\\
		Dr. Joao Carlos Pinto Barros\\
		
		\vspace{1cm}
        \small
        Institut für Theoretische Physik\\
        ETH Zürich\\
        

    \end{center}


\end{titlepage}


\section{Introduction}

\section{Theoretical background}

\subsection{The $\mathds{Z}_2$ gauge theory and its Quantum Ising dual}

The Ising gauge theory, theory with a discrete $\mathds{Z}_2$ gauge symmetry in $2+1$-dimensions \cite{fradkin}

\begin{equation}
	H_{\mathds{Z}_2} = -g\sum_{\vec{x},j}\sigma^x_j(\vec{x}) - \frac{1}{g}\sum_{\vec{x}}\sigma_1^z(\vec{x})\sigma_2^z(\vec{x}+e_1)\sigma_1^z(\vec{x}+e_2)\sigma_2^z(\vec{x}),
\end{equation}

where $\vec{x}$ refers to a position on the lattice, $j=1,2$ the two possible directions of a link, $\sigma_j^{x/z}(\vec{x})$ are the Pauli matrices, living on the links of the lattice. The local operator

\begin{equation}
	Q(\vec{x}) \equiv \sigma_1^x(\vec{x})\sigma_1^x(\vec{x}-e_1)\sigma_2^x(\vec{x})\sigma_2^x(\vec{x}-e_2)
\end{equation}

commutes with $H_{\mathds{Z}_2}$. It generates local gauge transformations. One can check that $Q^2=\mathds{1}$, so that the local symmetry of our problem is indeed $\mathds{Z}_2$. The operator

\begin{equation}
	\tau^z(\vec{r}) = \prod_{(\vec{x},j) \text{ pierced by } \gamma(\vec{r})}\sigma_j^x(\vec{x}),
\end{equation}

called the magnetic charge, is a gauge invariant quantity. $\gamma$ is an open path on the dual lattice. Since

\begin{equation}
	\left\{W_p^2(\vec{r}),\tau_z^2(\vec{r})\right\} = 0\text{ and }W_p^2(\vec{r}) = \tau_z^2(\vec{r}) = \mathds{1},
\end{equation} 

one can identify $W_p$ with the Pauli matrice $\tau^x$ on  the dual lattice.

\begin{equation}
	H = -\sum_{i=1}^N\tau^z_i\tau^z_{i+1} - g\sum\tau^x_j.
\end{equation}

where $i,j$ run over the dual lattice sites.


\subsection{The classical mapping of the Quantum Ising Model}

Our transverse Ising Model is not diagonal in the eigen-basis of $S^z$ and requires a quantum treatment. Luckily for us, a combination of clever tricks allows us to map the transverse field Ising Hamiltonian in $d$ dimensions to a classical anisotropic Ising Hamiltonian in $d+1$ dimensions \cite{Chakrabarti}. Let us demonstrate this in the case of the $1-D$ Quantum Ising model for simplicity. Its Hamiltonian simply reads

\begin{equation}
	H = -J\sum_{i=1}^N S_i^zS_{i+1}^z - \Gamma\sum_{i=1}^N S_i^x.
\end{equation}

The thermodynamical properties of this system can all be derived through its partition function $Z = \mathrm{Tr}\exp(-\beta H)$, with $\beta=1/k_BT$. Everything starts with the Trotter formula \cite{trotter}:

\begin{equation}
	\exp\left(A_1+A_2\right) =  \lim_{M\to\infty}\left[\exp\left(A_1/M\right)\exp\left(A_2/M\right)\right]^M.
\end{equation}

Writing our Hamiltonian as $H = H_0 + V$, where $H_0$ is the spin-spin interaction and $V$ the action of the magnetic field, one can expand the partition function as follows (keeping the large $M$ limit implicit for neatness):

\begin{align*}
	Z &= \mathrm{Tr}e^{-\beta\left(H_0+V\right)}\\
	&= \mathrm{Tr}\left[e^{-\beta H_0/M}e^{-\beta V/M}\right]^M\\
	&= \sum_{\left\{S^1_1,\cdots,S^1_N\right\}}\bra{S^1_1,\cdots,S^1_N}e^{-\beta H_0/M}e^{-\beta V/M}\cdots\\
	&\hspace{3cm}\times e^{-\beta H_0/M}e^{-\beta V/M}\ket{S^1_1,\cdots,S^1_N},
\end{align*}

where $\ket{S_i^1} = \ket{\pm1}_i$ is an eigenstates of $S_i^z$. The sum runs over all possible configurations for the lattice. Now, between each pair of exponential we can insert the identity in the form

\begin{equation}
	\mathds{1} = \sum_{\left\{S^k_1,\cdots,S^k_N\right\}}\ket{S^k_1,\cdots,S^k_N}\bra{S^k_1,\cdots,S^k_N}.
\end{equation}

The $k$ index, which simply labels one complete set of eigenstates of $S_z$, can be regarded as an additional dimension to our lattice (often referred to as the Trotter dimension). The spin $S_i^k$ can be understood the spin living on the $(i,k)$ site of a 2 dimensional lattice. Then

\begin{equation}
	Z = \sum_{\left\{S\right\}}\prod_{k=1}^M\bra{S^k_1,\cdots,S^k_N}e^{-\beta H_0/M}e^{-\beta V/M}\ket{S^{k+1}_1,\cdots,S^{k+1}_N},
\end{equation}

where we introduced periodic boundary conditions in the Trotter dimension, $\ket{S^{M+1}_1,\cdots,S^{M+1}_N} = \ket{S^1_1,\cdots,S^1_N}$ so as to match the braket sandwiching of the trace. We have use the shorthand $\sum_{\left\{S\right\}} = \sum_{\left\{S_1^1,\cdots,S_N^1\right\}}\cdots\sum_{\left\{S_1^M,\cdots,S_N^M\right\}}$, which is just a sum over all possible configurations of our 2D lattice. Now, the $H_0$ exponential is diagonal in the $\ket{S^k_1,\cdots,S^k_N}$ basis, so that, using its hermiticity

\begin{equation}
	\bra{S^k_1,\cdots,S^k_N}e^{-\beta H_0/M} = \exp\left[\frac{\beta J}{M}\sum_{i=1}^N S_i^k S_{i+1}^k\right]\bra{S^k_1,\cdots,S^k_N}.
\end{equation}

The second exponential now reads

\begin{align*}
	\bra{S^k_1,\cdots,S^k_N}e^{-\beta V/M}\ket{S^{k+1}_1,\cdots,S^{k+1}_N} &= \bra{S^k_1,\cdots,S^k_N}\exp\left[\frac{\beta\Gamma}{M}\sum_{i=1}^N S_i^x\right]\ket{S^{k+1}_1,\cdots,S^{k+1}_N}\\
	&= \prod_{i=1}^N\bra{S_i^k}e^{\beta\Gamma S_i^x/M}\ket{S_i^{k+1}}.
\end{align*}

Here comes a trick, keeping in mind that $S^x$ just flips the spin and that $(S^x)^2=\mathds{1}$:

\begin{align*}
	\bra{S}e^{aS^x}\ket{S'} &= \bra{S}\left(\sum_{n=0}^\infty\frac{a^{2n}(S^x)^{2n}}{n!} + \sum_{n=0}^\infty\frac{a^{2n+1}(S^x)^{2n+1}}{n!}\right)\ket{S'}\\
	&= \cosh{a}\braket{S,S'} + \sinh{a}\bra{S}S^x\ket{S'}\\
	&= \cosh{a}\braket{S,S'} + \sinh{a}\braket{S,-S'}\\
	&= \begin{cases}
		&\cosh{a} \text{ if } S=S',\\
		&\sinh{a} \text{ if } S=-S'.
	\end{cases}
\end{align*}

Now, simply note that

\begin{align*}
	\left(\frac{1}{2}\sinh{2a}\right)^{1/2}\exp\left[\frac{SS'}{2}\ln\coth{a}\right] &= \sqrt{\sinh{a}\cosh{a}}\times
		\begin{cases}
			&\exp\left[\dfrac{1}{2}\ln\coth{a}\right] \text{ if } S=S'			\vspace{1mm}\\
			&\exp\left[-\dfrac{1}{2}\ln\coth{a}\right] \text{ if } S=-S'
		\end{cases}\\
	& = \sqrt{\sinh{a}\cosh{a}}\times
	\begin{cases}
		&\sqrt{\cosh{a}/\sinh{a}} \text{ if } S=S'\\
		&\sqrt{\sinh{a}/\cosh{a}} \text{ if } S=-S'
	\end{cases}\\
	&= \begin{cases}
		&\cosh{a} \text{ if } S=S'\\
		&\sinh{a} \text{ if } S=-S'
	\end{cases}\\
	&= \bra{S}e^{aS^x}\ket{S'}
\end{align*}

This was a long road, but we can finally put everything together.

\begin{align*}
	Z &= \sum_{\left\{S\right\}}\prod_{k=1}^M\prod_{i=1}^N\exp\left[\frac{\beta J}{M}S_i^kS_{i+1}^k\right]\left(\frac{1}{2}\sinh{2\beta\Gamma/M}\right)^{1/2}\exp\left[\frac{S_i^kS_i^{k+1}}{2}\ln\coth(\beta\Gamma/M)\right]\\
	&= \left(\frac{1}{2}\sinh(2\beta\Gamma/M)\right)^{MN/2}\sum_{\left\{S\right\}}\exp\left[-\frac{\beta}{M}\left(-J\sum_{i=1}^N\sum_{k=1}^M S_i^k S_{i+1}^k - \frac{M}{2\beta}\ln\coth\left(\dfrac{\beta\Gamma}{M}\right)\sum_{i=1}^N\sum_{k=1}^M S_i^k S_i^{k+1}\right)\right]\\
	&= C\hspace{1mm}\mathrm{Tr}\left[e^{-\beta_\mathrm{cl}H_\mathrm{eff}}\right],
\end{align*}

with the classical temperature $\beta_\mathrm{cl} = \beta/M$. Let us also abbreviate $K_M = \dfrac{1}{2\beta}\ln\coth\left(\dfrac{\beta\Gamma}{M}\right)$. In the large $M$ limit, the prefactor vanishes, so that it drops out of any physical observables (obtained by differentiating the logarithm of the partition function). One recognizes in $H_\mathrm{eff}$ the Hamiltonian of a classical (since the $S_i^k$ are just numbers) anisotropic Ising model without any magnetic field. The weird thing is that the coupling constant of the spin-spin interactions depend on the lattice size in the Trotter dimension and even on the temperature for the interactions along the Trotter dimension. The quantum Ising model dual to our $\mathds{Z}_2$ gauge Hamiltonian is two dimensional, the above result therefore needs to be generalized to higher dimensions. The quantum Ising model in 2D

\begin{equation}
	H = -J\sum_{i,j}S_{i,j}\left(S^z_{i+1,j} + S^z_{i,j+1}\right) -\Gamma\sum_{i,j}S^x_{i,j}
\end{equation}

can be mapped to the 3D classical anisotropic Ising model

\begin{equation}
	H_\mathrm{eff} = -\sum_{k=1}^M\sum_{i,j}\left[J S_{i,j,k}\left(S_{i+1,j,k} + S_{i,j+1,k}\right) + K_M S_{i,j,k}S_{i,j,k+1}\right]
\end{equation}


\section{MCMC-MH}


\begin{tcolorbox}[title=Metropolis-Hasting algorithm]
	
	\begin{enumerate}
		
		\item Select initial value $\theta_0$.
		
		\item For $i\in\left\{1,\cdots,N_\mathrm{sample}\right\}$:
			\begin{itemize}
				\item Draw candidate $\theta^*$ from proposal distribution $q(\theta^*|\theta_{i-1})$.
				
				\item Compute $\alpha = \dfrac{g(\theta^*)}{g(\theta_{i-1})}\dfrac{q(\theta_{i-1}|\theta^*)}{q(\theta^*|\theta_{i-1})}$.
				
				\item If $\alpha \geq 1$ accept $\theta^*$ by setting $\theta_i\leftarrow\theta^*$,\\
				If $0 < \alpha <1$ accept $\theta^*$ with probability $\alpha$.
			\end{itemize}
		
	\end{enumerate}
	
\end{tcolorbox}


\clearpage\printbibliography

\end{document}